\usepackage[style=philosophy-modern,
refsection=chapter,
sortcites=true,
lowscauthors=true,
scauthorsbib=true,
annotation=true,
backend=biber,
labeldateparts=true,
backref=true,
useprefix=true,
citereset=chapter,
indexing=true,
relatedformat=brackets,
publocformat=loccolonpub,
volnumformat=strings,
latinemph=true,
inbeforejournal=true,
shorthandintro=true,
texencoding=utf8,
bibencoding=utf8,
uniquelist=minyear]{biblatex}

%GENERAR LOS INDICES, DESACTIVAMOS LAS OPCIONES PARA TITULOS
\renewbibmacro*{bibindex}{%
	\ifbibindex
	{\indexnames{author}%
		\indexnames{editor}%
		\indexnames{translator}%
		\indexnames{commentator}}
	{}}

\renewbibmacro*{citeindex}{%
	\ifciteindex
	{\indexnames{author}%
		\indexnames{editor}%
		\indexnames{translator}%
		\indexnames{commentator}}
	{}}

% CAMBIAMOS URL PARA QUE APAREZCA LA LEYENDA EN LINEA
\makeatletter
\newrobustcmd{\mkbiblege}[1]{%
	\begingroup
	\blx@blxinit
	\blx@setsfcodes
	<#1>
	\endgroup}
\makeatother

\DeclareFieldFormat{url}{\bibstring{url}\space\mkbiblege{\url{#1}}}

% REDEFINIR EL FORMATO DE CITADO EN PÁGINA 00
% \DeclareFieldFormat{pagerefformat}{\mkbibparens{{\color{red}\mkbibemph{#1}}}}
\renewbibmacro*{pageref}{%
	\iflistundef{pageref}
	{}
	{\printtext[pagerefformat]{%
			\ifnumgreater{\value{pageref}}{1}
			{\bibstring{backrefpages}\ppspace}
			{\bibstring{backrefpage}\ppspace}%
			\printlist[pageref][-\value{listtotal}]{pageref}}}}

% DECLARO UNA ALIAS PARA TRATAR MOVIE COMO MISC
\DeclareBibliographyAlias{movie}{misc}

% PONE PUNTO Y COMA ENTRE NOMBRE DE VARIOS AUTORES
\renewcommand*{\multinamedelim}{\addsemicolon\space}

% Cambia la tipografia de la bibliografia
\renewcommand*{\annotationfont}{\small\sf}
\renewcommand*{\bibfont}{\small}
% Cambia la longitud y el grosor
\renewcommand*{\bibnamedash}{\rule[.4ex]{1.5pc}{0.5pt}\hspace{1.5pc}}
%modifico el valor por default (4) de indentacion del año
\setlength{\bibhang}{3pc}

\setlength{\bibhang}{3\parindent}%modifico el valor por default (4) de indentacion del año

\defcounter{biburlnumpenalty}{3000}
\defcounter{biburlucpenalty}{6000}
\defcounter{biburllcpenalty}{9000}

%% REDISEÑO PARA OBTENER MODO LARGO EN ALGUNAS ENTRADAS
\NewBibliographyString{organizer}
\NewBibliographyString{organizers}
\NewBibliographyString{byorganizer}
\NewBibliographyString{coordinator}
\NewBibliographyString{coordinators}
\NewBibliographyString{bycoordinator}
\NewBibliographyString{direction}
\NewBibliographyString{directions}
\NewBibliographyString{bydirection}
\NewBibliographyString{continuator}
\NewBibliographyString{bycontinuator}
\NewBibliographyString{reviser}
\NewBibliographyString{byreviser}
\NewBibliographyString{redactor}
\NewBibliographyString{byredactor}
\NewBibliographyString{compiler}
\NewBibliographyString{bycompiler}
\NewBibliographyString{collaborator}
\NewBibliographyString{bycollaborator}


\NewBibliographyString{origpubbare}
\NewBibliographyString{documentjob}
\NewBibliographyString{dirigida}
\NewBibliographyString{dirigidas}
\NewBibliographyString{bydirigida}
\NewBibliographyString{escrita}
\NewBibliographyString{escritas}
\NewBibliographyString{byescrita}
\NewBibliographyString{elenco}
\NewBibliographyString{elencos}
\NewBibliographyString{byelenco}
\NewBibliographyString{mimeograph}
\NewBibliographyString{photocopy}
\NewBibliographyString{digitalprinting}
\NewBibliographyString{edauthor}
\NewBibliographyString{sq}
\NewBibliographyString{sqq}
\NewBibliographyString{bachelorthesis}
\NewBibliographyString{candthesis}
\NewBibliographyString{engineerthesis}
\NewBibliographyString{specializationthesis}
\NewBibliographyString{fieldwork}
\NewBibliographyString{finalessay}
\NewBibliographyString{finalmonograph}
\NewBibliographyString{finalproject}
\NewBibliographyString{gradothesis}
\NewBibliographyString{integrativefinalwork}
\NewBibliographyString{magisterthesis}
\NewBibliographyString{masterfinalwork}
\NewBibliographyString{mathesis}
\NewBibliographyString{phdthesis}
\NewBibliographyString{postgradthesis}
\NewBibliographyString{teacherthesis}
\NewBibliographyString{researchproject}
\NewBibliographyString{residencyreport}
\NewBibliographyString{shortthesis}
\NewBibliographyString{technicalreport}
\NewBibliographyString{undergraduatethesis}

\newcommand*{\sq}{\bibstring{sq}}
\newcommand*{\sqq}{\bibstring{sqq}}

\DefineBibliographyStrings{spanish}{%
sq = {~y~sig\adddot},
sqq = {~y~ss\adddot},
bachelorthesis = {Tesis de Licenciatura},
candthesis = {Trabajo candidato a Tesis},
engineerthesis = {Tesis de Ingeniería},
specializationthesis = {Tesis de Especialización},
fieldwork = {Trabajo de Campo},
finalessay = {Ensayo Final},
finalmonograph = {Monografía Final de Carrera},
finalproject = {Proyecto Final},
gradothesis = {Tesis de Grado},
integrativefinalwork = {Trabajo Integrador Final (TIF)},
magisterthesis = {Tesis de Magíster},
masterfinalwork = {Trabajo Final de Maestría},
mathesis = {Tesis de Maestría},
phdthesis = {Tesis Doctoral},
postgradthesis = {Tesis de Posgrado},
teacherthesis = {Tesis de Profesorado},
researchproject = {Proyecto de Investigación},
residencyreport = {Informe Final de Residencia},
shortthesis = {Tesina},
technicalreport = {Trabajo Final Técnico},
undergraduatethesis = {Trabajo Final de Grado},
inpreparation    = {en preparación},
submitted        = {enviado},
forthcoming      = {de próxima aparición},
inpress          = {en imprenta},
prepublished     = {previamente publicado},
digitalprinting  = {impresión digital},
photocopy        = {fotocopia},
mimeograph       = {mimeo},
edauthor         = {edición de autor},
dirigida         = {dirección},
dirigidas        = {dirección},
bydirigida       = {Dirección de},
escrita          = {escrita},
escritas         = {escrita},
byescrita        = {escrita por},
elenco           = {actúan:},
elencos          = {actúan:},
byelenco         = {actúan:},
part             = {tomo},
january          = {enero},
february         = {febrero},
march            = {marzo},
april            = {abril},
may              = {mayo},
june             = {junio},
july             = {julio},
august           = {agosto},
september        = {septiembre},
october          = {octubre},
november         = {noviembre},
december         = {diciembre},
see              = {véase},
seealso          = {véase también},
backrefpage      = {re\-fe\-ren\-cia ci\-ta\-da en pági\-na},
backrefpages     = {re\-fe\-ren\-cia ci\-ta\-da en pági\-nas},
seenote          = {véase nota},
quotedin         = {citado en},
idem             = {ídem},
idemsf           = {ídem},
idemsm           = {ídem},
idemsn           = {ídem},
idempf           = {ídem},
idempm           = {ídem},
idempn           = {ídem},
idempp           = {ídem},
ibidem           = {\emph{ibidem}},
prepublished     = {previamente publicado},
nodate           = {s/f},
withcommentator  = {con comentario de},
withannotator    = {con notas de},
withintroduction = {con introdución de},
withforeword     = {con prólogo de},
withafterword    = {con epílogo de},
collaborator     = {colaborador},
bycollaborator   = {con colaboración de},
continuator      = {continuador},
bycontinuator    = {continuado por},
andothers        = {\emph{et al\adddot}},
organizer        = {org\adddot},
organizers       = {orgs\adddot},
byorganizer      = {org\adddot\addspace por},
coordinator      = {coord\adddot},
coordinators     = {coords\adddot},
bycoordinator    = {coord\adddot\addspace por},
direction        = {dir\adddot},
directions       = {dirs\adddot},
bydirection      = {dir\adddot\addspace por},
documentjob      = {documento de trabajo},
urlseen          = {visitado el\addspace},
translationof    = {trad\adddot\addspace de},
translationas    = {original publicado en},
url              = {recuperado de},
origpubbare      = {orig\adddotspace pub\adddotspace},
newseries        = {nueva época},
oldseries        = {antigua época},
byeditorfo       = {ed\adddotspace y pról\adddotspace por},
byeditorco       = {ed\adddotspace y com\adddotspace por},
bytranslatorfo   = {traducido \lbx@lfromlang\ y prologado por},
pagetotal        = {págs},
pagetotals       = {págs},
techreport       = {informe técnico},
resreport        = {reporte de investigación},
file             = {archivo},
patent           = {patente},
patentde         = {patente alemana},
patenteu         = {patente europea},
patentfr         = {patente francesa},
patentuk         = {patente británica},
patentus         = {patente estadounidense},
patreq           = {solicitud de patente},
patreqde         = {solicitud de patente alemana},
patreqeu         = {solicitud de patente europea},
patreqfr         = {solicitud de patente francesa},
patrequk         = {solicitud de patente británica},
patrequs         = {solicitud de patente estadounidense},
compiler       = {compilador},
bycompiler       = {comp\adddotspace por},
redactor         = {redactor},
byredactor       = {redactado por},
reviser          = {revisor},
byreviser        = {revisado por},
issue            = {n\sptext{o}},
number           = {n\sptext{o}},
volume           = {vol\adddot},
volumes          = {vols\adddot},
}

% corrección para año original
\makeatletter
\renewbibmacro*{transorigstring}{%
	\iffieldundef{reprinttitle}%
	{\printtext{\ifdefstring{\bbx@origfields}{origed}
			{\bibstring{origpubbare}}%
			{\bibstring{translationas}}}\nopunct}%
	{\printtext{\bibstring{reprint}}}\nopunct}
\makeatother

% instrucción que me permite crear titulo y año como cita
\DeclareCiteCommand{\titleyear}
{\usebibmacro{prenote}}
{%
	\bibhyperref{%
		\usebibmacro{cite:label}%
		\iffieldundef{year}
		{}%
		{~(\printfield{year})}%
	}%
}
{\multicitedelim}
{\usebibmacro{postnote}}

% instrucción que me permite crear citeyearpar para obtener más de una cita
\DeclareCiteCommand{\citeyearpar}
{}
{%
	\ifnumequal{\value{citetotal}}{1}
	{(\bibhyperref{\printdate})}
	{%
		\ifnumequal{\value{citecount}}{1}
		{(\bibhyperref{\printdate}}
		{%
			\ifnumequal{\value{citecount}}{\value{citetotal}}
			{\addcomma\addspace\bibhyperref{\printdate})}
			{\addcomma\addspace\bibhyperref{\printdate}}%
		}%
	}%
}
{}
{}
